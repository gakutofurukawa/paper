\section{問題設定}

\subsection{対象とする問題}
% サンプル文章(実際の論文では削除してください)
本研究では、〇〇における△△の問題を扱う。
具体的には、入力として□□が与えられたとき、××を出力する問題を考える。

\subsection{問題の定式化}
% サンプル文章(実際の論文では削除してください)
問題を数式で表すと以下のようになる。
\begin{equation}
y = f(x; \theta)
\end{equation}
ここで、$x$は入力、$y$は出力、$\theta$はパラメータである。

\subsection{評価基準}
% サンプル文章(実際の論文では削除してください)
提案手法の性能は、以下の指標で評価する。
\begin{itemize}
\item 精度(Accuracy)
\item 処理時間(Processing Time)
\item メモリ使用量(Memory Usage)
\end{itemize}

\subsection{制約条件}
% サンプル文章(実際の論文では削除してください)
本研究では、以下の制約条件を設定する。
\begin{itemize}
\item 制約1:リアルタイム処理が可能であること
\item 制約2:汎用的なハードウェアで動作すること
\end{itemize}
