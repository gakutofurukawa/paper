\section{考察}

\subsection{実験結果の分析}
% サンプル文章(実際の論文では削除してください)
実験1の結果から、提案手法が〇〇において有効であることが示された。
特に、△△の場合に精度が大きく向上している点が注目される。
これは、手法2で導入した□□が効果的に機能したためと考えられる。

\subsection{処理速度の改善要因}
% サンプル文章(実際の論文では削除してください)
処理速度が2倍向上した要因として、以下の2点が挙げられる。
\begin{itemize}
\item 前処理の最適化により、計算量が削減された
\item 並列処理を導入したことで、マルチコア環境での性能が向上した
\end{itemize}

\subsection{既存研究との比較}
% サンプル文章(実際の論文では削除してください)
既存研究\cite{sample_ref1}と比較すると、本手法は精度で同等、処理速度で優位である。
ただし、メモリ使用量については既存手法より約20\%増加している。

\subsection{限界と今後の課題}
% サンプル文章(実際の論文では削除してください)
本研究の限界として、以下の点が挙げられる。
\begin{itemize}
\item 特定のドメインに特化しているため、汎用性に課題がある
\item 大規模データセットでの検証が不十分である
\end{itemize}
今後は、これらの課題に取り組む必要がある。

\subsection{本章まとめ}
% サンプル文章(実際の論文では削除してください)
本章では、実験結果の考察を行い、提案手法の有効性と課題を明らかにした。
