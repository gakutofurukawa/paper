\section{提案手法}

\subsection{システム概要}
% サンプル文章(実際の論文では削除してください)
本研究では、〇〇を実現するために△△システムを提案する。
図\ref{fig:system_overview}にシステムの全体像を示す。
% \begin{figure}[h]
% \centering
% \includegraphics[width=0.8\linewidth]{image/system_overview.png}
% \caption{システム概要}
% \label{fig:system_overview}
% \end{figure}

\subsection{手法1}
% サンプル文章(実際の論文では削除してください)

\subsubsection{目的}
手法1の目的は、□□を効率的に処理することである。

\subsubsection{詳細}
具体的には、以下のアルゴリズムを用いる。
\begin{enumerate}
\item ステップ1:入力データの前処理
\item ステップ2:特徴抽出
\item ステップ3:分類処理
\end{enumerate}

\subsection{手法2}
% サンプル文章(実際の論文では削除してください)
手法2では、××を改善するために△△を導入する。
これにより、処理速度が約2倍向上することが期待される。

\subsection{実装}
% サンプル文章(実際の論文では削除してください)
提案手法はPythonで実装した。
主要なライブラリとして、NumPy、SciPy、scikit-learnを使用している。

\subsection{まとめ}
% サンプル文章(実際の論文では削除してください)
本章では、提案手法の詳細について説明した。
次章では、この手法の有効性を実験により検証する。
